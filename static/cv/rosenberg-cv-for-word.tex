\documentclass[14,]{article}
\usepackage{lmodern}
\usepackage{amssymb,amsmath}
\usepackage{ifxetex,ifluatex}
\usepackage{fixltx2e} % provides \textsubscript
\ifnum 0\ifxetex 1\fi\ifluatex 1\fi=0 % if pdftex
  \usepackage[T1]{fontenc}
  \usepackage[utf8]{inputenc}
\else % if luatex or xelatex
  \ifxetex
    \usepackage{mathspec}
  \else
    \usepackage{fontspec}
  \fi
  \defaultfontfeatures{Ligatures=TeX,Scale=MatchLowercase}
\fi
% use upquote if available, for straight quotes in verbatim environments
\IfFileExists{upquote.sty}{\usepackage{upquote}}{}
% use microtype if available
\IfFileExists{microtype.sty}{%
\usepackage{microtype}
\UseMicrotypeSet[protrusion]{basicmath} % disable protrusion for tt fonts
}{}
\usepackage[margin=1in]{geometry}
\usepackage{hyperref}
\hypersetup{unicode=true,
            pdftitle={Joshua M. Rosenberg},
            pdfborder={0 0 0},
            breaklinks=true}
\urlstyle{same}  % don't use monospace font for urls
\usepackage{graphicx,grffile}
\makeatletter
\def\maxwidth{\ifdim\Gin@nat@width>\linewidth\linewidth\else\Gin@nat@width\fi}
\def\maxheight{\ifdim\Gin@nat@height>\textheight\textheight\else\Gin@nat@height\fi}
\makeatother
% Scale images if necessary, so that they will not overflow the page
% margins by default, and it is still possible to overwrite the defaults
% using explicit options in \includegraphics[width, height, ...]{}
\setkeys{Gin}{width=\maxwidth,height=\maxheight,keepaspectratio}
\IfFileExists{parskip.sty}{%
\usepackage{parskip}
}{% else
\setlength{\parindent}{0pt}
\setlength{\parskip}{6pt plus 2pt minus 1pt}
}
\setlength{\emergencystretch}{3em}  % prevent overfull lines
\providecommand{\tightlist}{%
  \setlength{\itemsep}{0pt}\setlength{\parskip}{0pt}}
\setcounter{secnumdepth}{0}
% Redefines (sub)paragraphs to behave more like sections
\ifx\paragraph\undefined\else
\let\oldparagraph\paragraph
\renewcommand{\paragraph}[1]{\oldparagraph{#1}\mbox{}}
\fi
\ifx\subparagraph\undefined\else
\let\oldsubparagraph\subparagraph
\renewcommand{\subparagraph}[1]{\oldsubparagraph{#1}\mbox{}}
\fi

%%% Use protect on footnotes to avoid problems with footnotes in titles
\let\rmarkdownfootnote\footnote%
\def\footnote{\protect\rmarkdownfootnote}

%%% Change title format to be more compact
\usepackage{titling}

% Create subtitle command for use in maketitle
\newcommand{\subtitle}[1]{
  \posttitle{
    \begin{center}\large#1\end{center}
    }
}

\setlength{\droptitle}{-2em}

  \title{Joshua M. Rosenberg}
    \pretitle{\vspace{\droptitle}\centering\huge}
  \posttitle{\par}
    \author{}
    \preauthor{}\postauthor{}
    \date{}
    \predate{}\postdate{}
  

\begin{document}
\maketitle

PDF version:

\emph{Curriculum Vitae}

Assistant Professor, STEM Education\\
Department of Theory and Practice in Teacher Education\\
The University of Tennessee, Knoxville\\
420 Claxton, 1122 Volunteer Blvd., Knoxville, TN 37996\\
865-974-5973 \textbar{}
\href{mailto:jmrosenberg@utk.edu}{\nolinkurl{jmrosenberg@utk.edu}}
\textbar{} \url{http://joshuamrosenberg.com}

\subsection{Research Interests}\label{research-interests}

Science education, data science education, quantitative and
computational research methods, educational technology

\subsection{Education}\label{education}

2018, PhD, Educational Psychology \& Educational Technology\\
Michigan State University

2016, Graduate Certificate, Science Education\\
Michigan State University

2012, MA, Education\\
Michigan State University

2010, Teacher Licensure Program\\
University of North Carolina, Asheville

2010, BS, Biology\\
University of North Carolina, Asheville

\subsection{Professional Experience}\label{professional-experience}

2018-present, Assistant Professor, STEM Education\\
University of Tennessee, Knoxville

2012-2018, Graduate Research and Teaching Assistant\\
Michigan State University

\subsection{Related (K-12 Teaching)
Experience}\label{related-k-12-teaching-experience}

2010-2012, Science Teacher (Biology and Earth Science)\\
Shelby High School, Shelby, NC

2009-2010, Science Teacher Intern (Biology and Chemistry)\\
C.D. Owen High School, Swannanoa, NC

\subsection{Publications}\label{publications}

\subsubsection{Peer-Reviewed Journal
Articles}\label{peer-reviewed-journal-articles}

Blondel, D. V., Sansone, A., Rosenberg, J. M., Yang, B. W.,
Linennbrink-Garcia, L., Schwarz-Bloom, R. D. (in press). Development of
an online experiment platform (Rex) for high school biology.
\emph{Journal of Formative Design for Learning}.

Henriksen, D., Mehta, R., \& Rosenberg, J. M. (in press). Supporting a
creatively focused technology fluent mindset among educators: Survey
results from a five-year inquiry into teachers' confidence in using
technology. \emph{Journal of Technology and Teacher Education}

Xu, R., Frank, K. A., Maroulis, S., \& Rosenberg, J. M. (in press).
Konfound: A Stata module to quantify robustness of causal inferences.
\emph{The Stata Journal}. \url{https://www.stata-journal.com/}
(\emph{Nb. Software-related publication})

Rosenberg, J. M., \& Lawson, M. J. (2019). An investigation of students'
use of a computational science simulation in an online high school
physics class, \emph{Education Sciences, 9}(49), 1-19.
\url{https://www.mdpi.com/2227-7102/9/1/49}

Rosenberg, J. M., Beymer, P. N., Anderson, D. J., \& Schmidt, J. A.
(2018). tidyLPA: An R package to easily carry out Latent Profile
Analysis (LPA) using open-source or commercial software. \emph{Journal
of Open Source Software, 3}(30), 978.
\url{https://doi.org/10.21105/joss.00978} (\emph{Nb. Software-related
publication})

Greenhalgh, S. P., Staudt Willet, K. B., Rosenberg, J. M., \& Koehler,
M. J. (2018). Tweet, and we shall find: Using digital methods to locate
participants in educational hashtags. \emph{TechTrends, 62}(5), 501-508.
\url{https://doi.org/10.1007/s11528-018-0313-6}

Beymer, P. N., Rosenberg, J. M., Schmidt, J. A., \& Naftzger, N. (2018).
Examining relationships between choice, affect, and engagement in
out-of-school time STEM programs. \emph{Journal of Youth and
Adolescence, 47}(6), 1178-1191.
\url{https://doi.org/10.1007/s10964-018-0814-9}

Akcaoglu, M., Rosenberg, J. M., Ranellucci, J., \& Schwarz, C. V.
(2018). Outcomes from a self-generated utility value intervention on
fifth and sixth-grade students' value and interest in science.
\emph{International Journal of Educational Research, 87}, 67-77.
\url{https://www.sciencedirect.com/science/article/pii/S0883035517308492}

Schmidt, J. A., Rosenberg, J. M., \& Beymer, P. (2018). A
person-in-context approach to student engagement in science: Examining
learning activities and choice. \emph{Journal of Research in Science
Teaching, 55}(1), 19-43. \url{https://dx.doi.org/10.1002/tea.21409}
(\emph{Nb. This article was recognized as one of the 20 most-downloaded
articles in JRST between June, 2016 and June, 2018})

Rosenberg, J.M., Greenhalgh, S.P., Graves Wolf, L. \& Koehler, M.J.
(2017). Strategies, use, and impact of social media for supporting
teacher community within professional development: The case of one urban
STEM program. \emph{Journal of Computers in Mathematics and Science
Teaching, 36}(3), 255-267.
\url{https://www.learntechlib.org/primary/p/180387/}

Koehler, M. J., Greenhalgh, S. P., Rosenberg, M. J., \& Keenan, S.
(2017). What the tech is going on with digital teaching portfolios?
Using the TPACK framework to analyze teachers' technological
understanding. \emph{Journal of Technology and Teacher Education, 25},
31-59. \url{http://www.learntechlib.org/p/173346/}

Rosenberg, J. M., Greenhalgh, S. P., Koehler, M. J., Hamilton, E., \&
Akcaoglu, M. (2016). An investigation of State Educational Twitter
Hashtags (SETHs) as affinity spaces. \emph{E-Learning and Digital Media,
13}(1-2), 24-44. \url{http://dx.doi.org/10.1177/2042753016672351}

Greenhalgh, S. P., Rosenberg, J. M., \& Wolf, L. G. (2016). For all
intents and purposes: Twitter as a foundational technology for teachers.
\emph{E-Learning and Digital Media, 13}(1-2), 81-98.
\url{http://dx.doi.org/10.1177/2042753016672131}

Hamilton, E., Rosenberg, J. M., \& Akcaoglu, M. (2016). Examining the
Substitution Augmentation Modification Redefinition (SAMR) model for
technology integration. \emph{Tech Trends, 60}, 433-441.
\url{http://dx.doi.org/10.1007/s11528-016-0091-y}

Rosenberg, J. M., Terry, C. A., Bell, J., Hiltz, V., \& Russo, T.
(2016). Design guidelines for graduate program social media use.
\emph{Tech Trends, 2}, 167-175.
\url{http://dx.doi.org/10.1007/s11528-016-0023-x}

Rosenberg, J. M., \& Koehler, M. J. (2015). Context and Technological
Pedagogical Content Knowledge (TPACK): A systematic review.
\emph{Journal of Research on Technology in Education, 47}, 186-210.
\url{http://dx.doi.org/10.1080/15391523.2015.1052663}

\subsubsection{Book Chapters}\label{book-chapters}

Rosenberg, J. M., Lawson, M. A., Anderson, D. J., \& Rutherford, T.
(forthcoming). In E. Romero-Hall (Ed.), \emph{Research Methods in
Learning Design \& Technology.}, Data science dilemmas for teaching
research methods in learning design and technology. Routledge: New York,
NY.

Greenhalgh, S. P., Staudt Willet, B., Rosenberg, J. M., \& Koehler, M.
J. (forthcoming). In E. Romero-Hall (Ed.), \emph{Research Methods in
Learning Design \& Technology.}, Lessons learned from applying Twitter
research methods to educational technology phenomena. Routledge: New
York, NY.

Dai, T., Rosenberg, J. M., \& Lawson, M. A. (forthcoming). Data
representation. In T. L. Good \& M. McCaslin (\emph{Eds.}), Educational
Psychology Section; D. Fisher (\emph{Ed.}), Routledge Encyclopedia of
Education (Online). Taylor \& Francis: New York, NY.

Eidelman, R., Rosenberg, J. M. ,\& Shwartz, Y. (in press). E-Learning in
chemistry education: Self-regulated learning in a virtual classroom. In
D. Ifenthaler, M. Spector, P. Isafas, \& S. Sergis (Eds), \emph{Learning
technologies for transforming teaching, learning and assessment at large
scale}. Berlin, Germany: Springer.

Herring, M., Koehler, M. J., Mishra, P., Rosenberg, J. M., \& Teske, J.
(2016). Introduction to the 2nd edition of the TPACK handbook. In M.
Herring, M. J. Koehler, \& P. Mishra (Eds.), \emph{Handbook of
Technological Pedagogical Content Knowledge (TPACK) for educators} (2nd
ed., pp.~1-8). New York, NY: Routledge.

Keenan, S., Rosenberg, J. M., Greenhalgh, S. P. \& Koehler, M. J.
(2016). Examining teachers' technology use through digital portfolios.
In L. Liu \& D. C. Gibson (Eds.), \emph{Research highlights in
technology and teacher education 2016} (pp.~53-60). Chesapeake, VA:
Association for the Advancement of Computing in Education.

Phillips, M., Koehler, M. J. \& Rosenberg, J. M. (2016). Considering
context: Teachers' TPACK development and enactment. In L. Liu \& D. C.
Gibson (Eds.), \emph{Research highlights in technology and teacher
education} (pp.~197-204). Chesapeake, VA: Association for the
Advancement of Computing in Education.

Rosenberg, J. M., \& Koehler, M. J. (2015). \emph{Context and teaching
with technology in the digital age}. In M.L. Niess \& H. Gillow-Wiles
(Eds.), Handbook of research on teacher education in the digital age
(pp.~440-465). Hershey, PA: IGI Global.

Rosenberg, J. M., Greenhalgh, S. P., \& Koehler, M. J. (2015). A
performance assessment of teachers' TPACK using artifacts from digital
portfolios. In L. Liu \& D. C. Gibson, \emph{Research highlights in
technology and teacher education 2015}. Waynesville, NC: Association for
the Advancement of Computing in Education (AACE).

Koehler, M. J., Mishra, P., Akcaoglu, M., \& Rosenberg, J. M. (2013).
Technological pedagogical content knowledge for teachers and teacher
educators. In N. Bharati and S. Mishra (Eds.), \emph{ICT integrated
teacher education: A resource book} (pp.~1-8). Commonwealth Educational
Media Center for Asia, New Delhi, India.

\subsubsection{Conference Proceedings
Papers}\label{conference-proceedings-papers}

Carpenter, J., Rosenberg, J. M., Dousay, T., Romero-Hall, E., Trust, T.,
Kessler, A., Phillips, M., Morrison, S., Fischer, C. \& Krutka, D.
(2019). What do teacher educators think of teacher education technology
competencies?. In K. Graziano (Ed.), Proceedings of Society for
Information Technology \& Teacher Education International Conference
(pp.~796-801). Las Vegas, NV, United States: Association for the
Advancement of Computing in Education (AACE). Retrieved April 18, 2019
from \url{https://www.learntechlib.org/primary/p/207735/}.

Peterson, A., Freer, D., \& Rosenberg, J. M. (2017). Interacting with
purpose: What is the difference between face-to-face and online student
relationships in a combined program? In Proceedings of Society for
Information Technology \& Teacher Education International Conference
2016 (pp.~3411-3414). Austin, TX: Association for the Advancement of
Computing in Education. Retrieved from
\url{https://www.learntechlib.org/p/177955/}

Krist, C., \& Rosenberg, J. M. (2016). Finding patterns in and refining
characterizations of students' epistemic cognition: A computational
approach. In Looi, C.-K., Polman, J., Cress, U., \& Reimann, P. (Eds.),
\emph{Transforming Learning, Empowering Learners: The International
Conference of the Learning Sciences Proceedings} 2016 (Vol. 2,
pp.~1223-1224). Singapore, Singapore: ICLS.

\hangindent=2em Rosenberg, J. M., Koehler, M. J., Akcaoglu, M.,
Greenhalgh, S. P. \& Hamilton, E. (2016). State Educational Twitter
Hashtags: An introduction and research agenda. In \emph{Proceedings of
Society for Information Technology \& Teacher Education International
Conference 2016} (pp.~355-360). Chesapeake, VA: Association for the
Advancement of Computing in Education. Retrieved from
\url{http://www.editlib.org/p/171698}

Greenhalgh, S. P., Rosenberg, J. M. \& Wolf, L. G. (2016). For every
tweet there is a purpose: Twitter within (and beyond) an online graduate
program. In \emph{Proceedings of Society for Information Technology \&
Teacher Education International Conference 2016} (pp.~2044-2049).
Chesapeake, VA: Association for the Advancement of Computing in
Education (AACE). Retrieved from \url{http://www.editlib.org/p/171972}

Rosenberg, J. M., Greenhalgh, S. P. \& Koehler, M. J. (2015). A
performance assessment of teachers' TPACK using artifacts from digital
portfolios. In D. Slykhuis \& G. Marks (Eds.), \emph{Proceedings of
Society for Information Technology \& Teacher Education International
Conference 2015} (pp.~3390-3397). Chesapeake, VA: Association for the
Advancement of Computing in Education (AACE). Retrieved from
\url{http://www.editlib.org/p/150472}

Schwarz, C. V., Ke, L., Lee, M, \& Rosenberg, J. M. (2014). Developing
mechanistic explanations of phenomena: Case studies of two fifth grade
students' epistemologies in practice over time. In J. L. Polman, E. A.
Kyza, K. O'Neill, I. Tabak, W. R. Penuel, A. S. Jurow, . . . L. D'Amico
(Eds.), \emph{Learning and becoming in practice: The international
conference of the learning sciences (ICLS) 2014} (Vol. 1, pp.~182-189).
Boulder, CO: ISLS. \url{http://www.isls.org/icls2014/Proceedings.html}

Rosenberg, J. M., \& Koehler, M. (2014). Context and Technological
Pedagogical Content Knowledge: A content analysis. In M. Searson \& M.
Ochoa (Eds.), \emph{Proceedings of Society for Information Technology \&
Teacher Education International Conference 2014} (pp.~2412-2417).
Chesapeake, VA: AACE. Retrieved from
\url{http://www.editlib.org/p/131183}

Greenhalgh, S. P., Rosenberg, J. M., Zellner, A. \& Koehler, M. J.
(2014). Zen and the art of portfolio maintenance: Best practices in
course design for supporting long-lasting portfolios. In M. Searson \&
M. Ochoa (Eds.), \emph{Proceedings of Society for Information Technology
\& Teacher Education International Conference 2014} (pp.~1604-1610).
Chesapeake, VA: AACE. Retrieved from
\url{http://www.editlib.org/p/131027}

Rosenberg, J., Terry, C., Bell, J., Hiltz, V., Russo, T. \& The EPET
Social Media Council (2014). What we've got here is failure to
communicate: Social media best practices for graduate school programs.
In M. Searson \& M. Ochoa (Eds.), \emph{Proceedings of Society for
Information Technology \& Teacher Education International Conference
2014} (pp.~1210-1215). Chesapeake, VA: AACE. Retrieved from
\url{http://www.editlib.org/p/130949}

Rosenberg, J. (2013). Review of mobile device use policies in public
high schools. In R. McBride \& M. Searson (Eds.), \emph{Proceedings of
Society for Information Technology \& Teacher Education International
Conference 2013} (pp.~3774-3779). Chesapeake, VA: AACE. Retrieved from
\url{http://www.editlib.org/p/48698git}

\subsubsection{Editor-Reviewed
Publications}\label{editor-reviewed-publications}

Naftzger, N., Schmidt, J. A., Shumow, L., Beymer, P. N., \& Rosenberg,
J. M. (2019). \emph{Exploring the link between STEM activity leader
practice and youth engagement: Findings from the STEM IE study}.
Washington, DC: American Institutes for Research.

Mehta, R., Henriksen, D., \& Rosenberg, J. M. (2019). It's not about the
tools. \emph{Educational Leadership, 76}(5), 64-69. Retrieved from
\url{http://www.ascd.org/publications/educational-leadership/feb19/vol76/num05/It's-Not-About-the-Tools.aspx}

Vo, T., \& Rosenberg, J. M. (2018). Posts for the NARST Graduate Student
Resources blog {[}four blog posts{]}. \emph{NARST Graduate Student
Resources Blog}. Linked through this page:
\url{https://joshuamrosenberg.com/job-market-resources.html}

Rosenberg, J. M. (2018). Opportunities for engaging students in ``data
practices'' in online science classes. \emph{Michigan Virtual Learning
Research Institute Blog: Research, Policy, Innovation \& Networks}.
\url{https://mvlri.org/blog/opportunities-engaging-students-data-practices-online-science-classes/}

Rosenberg, J. M., \& Logan, C. W. (2017). Review of the book What's
Worth Teaching: Rethinking Curriculum in the Age of Technology, by A.
Collins. \emph{Teachers College Record}.
\url{http://www.tcrecord.org/Content.asp?ContentID=22173}

Phillips, M., Harris, J., Rosenberg, J. M., \& Koehler, M. J. (2017).
TPCK/TPACK research and development: Past, present, and future
directions. \emph{Australasian Journal of Educational Technology}.
\url{https://doi.org/10.14742/ajet.3907}

Rosenberg, J. M., \& Ranellucci, J. (2017). Student motivation in online
science courses: A path to spending more time on course and higher
achievement. \emph{Michigan Virtual Learning Research Institute Blog:
Research, Policy, Innovation \& Networks}.
\url{https://mvlri.org/blog/student-motivation-in-online-science-courses-a-path-to-spending-more-time-on-course-and-higher-achievement/}

\subsubsection{Journal Articles in
Submission}\label{journal-articles-in-submission}

Ranellucci, J., Rosenberg, J. M., \& Poitras, E. (accepted registered
report). Exploring pre-service teachers' use of technology: The
technology acceptance model and expectancy-value theory. \emph{Journal
of Computer Assisted Learning}.

Beymer, P. N., Rosenberg, J. M., \& Schmidt, J. A. (revise and
resubmit). Investigating the effects of interest and choice in
education: An experience sampling approach in high school science
classes.

Rosenberg, J. M., Schmidt, J. A., \& Koehler, M. J. (under review). How
youth experience work with data in summer STEM programs: Findings from
an experience sampling approach.

Krist, S., \& Rosenberg, J. M. (under review). Incremental refinement in
learners' epistemic considerations of generality in science over three
years: A computational grounded theory analysis.

Carpenter, J., Rosenberg, J. M., Dousay, T., Romero-Hall, E., Trust, T.,
Kessler, A., Phillips, M., Fischer, C., Morrison, S., \& Krutka, D.
(under review). Understanding teacher educator technology competencies:
findings and perspectives from a cross-disciplinary sample.

Schmidt, J. A., Beymer, P. N., Rosenberg, J. M., Naftzger, N., \&
Shumow, L. (under review). Experiences, activities, and personal
characteristics as predictors of engagement in STEM-focused summer
programs.

Anderson, D., Rowlew, B., Irvin, P. S., Rosenberg, J. M., \& Stegenga,
S. (under review). Evaluating content-related validity evidence using a
text-based, machine learning procedure.

\subsubsection{Working papers}\label{working-papers}

Rosenberg, J. M., Schwarz, C. V., \& Akcaoglu, M. (in preparation).
Learning to teach scientific modeling: A longitudinal case study of two
teachers' instructional practice.

Rosenberg, J. M., Reid, J., Koehler, M. J., Fischer, C., \& McKenna, T.
J. (in preparation). The roles of the Twitter hashtag \#NGSSchat in the
context of science education reform efforts.

Schmidt, J. A., Beymer, P. N., \& Rosenberg, J. M. (in preparation).
Experiences, activities, and personal characteristics as predictors of
engagement in STEM-focused summer programs.

Greenhalgh, S. P., Rosenberg, J. M., Staudt-Willet, B., Koehler, M. J.,
\& Akcaoglu, M. (in preparation). Timing is everything: Comparing
synchronous and asynchronous modes of Twitter for teacher professional
learning.

\subsubsection{Unpublished Manuscripts and
Pre-Prints}\label{unpublished-manuscripts-and-pre-prints}

Rosenberg, J. M. (2018). \emph{Context and Technological Pedagogical
Content Knowledge: An initial survey of teachers and validation data}.

Rosenberg, J. M. (2018). \emph{Understanding work with data in summer
STEM programs: An experience sampling method approach} (Doctoral
dissertation). Retrieved from Proquest Dissertations and Theses.
(Proquest No. 10747232)

\subsection{Grants}\label{grants}

2017-2020, Consultant, Profiles of science engagement: Broadening
participation by understanding individual and contextual influences
(\$499,927). National Science Foundation (PI: Jennifer Schmidt)

2019-2020, PI, Planting the seeds for computer science education in East
Tennessee through a research-practice partnership. (\$13,200). Community
Engaged Research Seed Program, University of Tennessee, Knoxville

2018-2019, Co-PI, Exploring how beginning elementary mathematics
teachers seek out resources through social media (\$8,820), Herman and
Rasiej Math Initiative at the University of Southern California (PI:
Stephen Aguilar)

2013, PI, Basic Biology for Everyone (\$2,000), Versal Foundation Grant

\subsection{Fellowships and Awards}\label{fellowships-and-awards}

2019, Open Publishing Support Fund, University of Tennessee Libraries
and Office of Research and Engagement, University of Tennessee,
Knoxville

2018, Foreign Travel Award, Office of Research and Engagement,
University of Tennessee, Knoxville (UTK)

2017, Delia Koo Global Travel Fellowship, Michigan State University
(MSU)

2017, Michigan Virtual Learning Research Institute Dissertation
Fellowship (\$1,500)

2017, Concord Consortium Data Science Educational Technology Fellowship
(\$1,000)

2017, Research Expenses Fellowship (\$3,250), MSU

2016, College of Education Alumni Fellowship (\$5,500), MSU

2015, Cotterman Family Endowment for Education Summer Research
Fellowship (\$6,000), MSU

2013, Massive Open Online Course Research and Development Fellowship
(\$1,000), MSU

\subsection{Awards}\label{awards}

2019, Finalist, Association for Science Teacher Education John C. Park
National Technology Leadership Institute Fellowship

2017, Council of Graduate Students Disciplinary Leadership Award,
Michigan State University

2016, Best Paper Award, TPACK SIG, SITE International Conference

2014, Outstanding Paper Award, Society for Information Technology and
Teacher Education International Conference

\subsection{Presentations}\label{presentations}

\subsubsection{Peer-Reviewed Conference
Presentations}\label{peer-reviewed-conference-presentations}

Rosenberg, J. M, Beymer, P. N., Houslay, T. M., \& Schmidt, J. A. (2019,
April). Using a multivariate, multi-level model to understand how
youths' in-the-moment engagement predicts changes in youths' interest.
In M. Bernacki, A. Kaplan, and L. Linnenbrink-Garcia (Chairs),
\emph{Embracing and modeling the complex dynamics of motivation and
engagement: Contextual, temporal, dynamic, and systematic}. Symposium
conducted at the Annual Meeting of the American Educational Research
Association, Toronto, CA.

Beymer, P. N., Schell, M. J., Alberts, K. M., Rosenberg, J. M., \&
Schmidt, J. A. (2019, April). \emph{Student engagement profiles in
formal and informal STEM learning settings}. Paper presented at the
Annual Meeting of the American Educational Research Association,
Toronto, Canada.

Schell, M. J., Beymer, P. N. Albert, K. M., Rosenberg, J. M., \&
Schmidt, J. A. (2019, April). \emph{Predictors of momentary student
engagement profiles in high school science classrooms}. Paper presented
at the Annual Meeting of the American Educational Research Association,
Toronto, Canada.

Reid, J., Rosenberg, J. M., Koehler, M. J., Fischer, C., \& McKenna, T.
J. (2019, March). \emph{An exploration of \#NGSSchat through social
network analysis}. Paper presented at the National Association for
Research in Science Teaching Annual International Conference, Baltimore,
MD.

Rosenberg, J. M., Reid, J., Koehler, M. J., Fischer, C., \& McKenna, T.
J. (2019, January). \emph{The roles of the Twitter hashtag \#NGSSchat in
the context of science education reform efforts}. Paper presented at the
Association for Science Teacher Education International Meeting,
Savannah, GA. (\emph{Nb. This paper was nominated for the ASTE John C.
Park National Technology Leadership Institute Fellowship})

Akcaoglu, M., Hodges, C. B., Rosenberg, J. M., \& Hilpert, J. (2018,
October). \emph{Factors impacting middle school students' interest,
efficacy, and utility value of programming}. Paper presented at the
Association for Educational Communications and Technology International
Convention 2018. Kansas City, MO.

Staudt Willet, K. B., Greenhalgh, S. P., Rosenberg, J. M., Koehler, M.
J. (2018, October). \emph{Won't you be my neighbor? How education
stakeholders use hyperlinks to build information neighborhoods on
Twitter}. Paper to be presented at the Association for Educational
Communications and Technology International Convention 2018. Kansas
City, MO.

Beymer, P. N., Rosenberg, J.M., \& Schmidt, J. A. (2018, April).
\emph{Investigating the effects of interest and choice: An experience
sampling approach}. Paper presented at the Annual Meeting of the
American Educational Research Association, New York, NY.

Greenhalgh, S. P., Staudt Willet, B., Rosenberg, J. M., Akcaoglu, M., \&
Koehler, M. J. (2018, April). \emph{Timing is everything: Comparing
synchronous and asynchronous modes of Twitter for teacher professional
learning}. Paper presented at the Annual Meeting of the American
Educational Research Association, New York, NY.

Rosenberg, J. M., Beymer, P. N., \& Schmidt, J. A. (2018, April).
\emph{How engagement during out-of-school time STEM programs predicts
changes in motivation in STEM}. In J. M. Rosenberg (Chair),
Data-intensive approaches to studying engagement in education: Exploring
their current potential. Paper presented at the Annual Meeting of the
American Educational Research Association, New York, NY.

Rosenberg, J. M., Lee, Y., Robinson, K. A., Ranellucci, J., Roseth, C.
J., \& Linnenbrink-Garcia, L. (2018, April). \emph{Patterns of
engagement in a flipped undergraduate class: Antecedents and outcomes}.
In L. Daniels \& A. Frenzel (Chairs), New empirical insights on what
energizes learners -- A session on emotions and engagement. Paper
presented at the Annual Meeting of the American Educational Research
Association, New York, NY.

Schmidt, J. A., Rosenberg, J.M., \& Beymer, P. N. (2018, April).
\emph{Experiences, activities, and personal characteristics as
predictors of interest and engagement in STEM-focused summer programs}.
Paper presented at the Annual Meeting of the American Educational
Research Association, New York, NY.

Shwartz, Y., Bayer, I., Bielik, T., Kolonich, A., Eidelman, R., Shwartz,
G., . . . Rosenberg, J. M. (2018, March). \emph{Graduate student
international collaboration for investigating science teachers'
professional learning}. Paper presented at the meeting of the National
Association for Research in Science Teaching, Atlanta, GA.

Yang, B. W., Blondel, D. V., Rosenberg, J. M., Sansone, A.,
Linennbrink-Garcia, L., Schwarz-Bloom, R. D. (2017, November). \emph{The
Rex virtual experiment platform: Design, implementation, and effects on
situational interest}. Poster presented at the Annual Meeting of the
Society for Neuroscience, Washington, DC.

Greenhalgh, S. P., Staudt Willet, K. B., Rosenberg, J. M., \& Koehler,
M. J. (2017, November). \emph{No accounting for theory? The case for an
affinity space approach to educational hashtag research}. Paper
presented at the Association for Educational Communications and
Technology International Convention 2017, Jacksonville, FL.

Greenhalgh, S. P., Rosenberg, J. M., \& Koehler, M. J. (2017, November).
\emph{Hide and go tweet: Comparing methods for locating educational
hashtag participants}. Paper presented at the Association for
Educational Communications and Technology International Convention 2017,
Jacksonville, FL.

Schmidt, J. A., Rosenberg, J. M., \& Beymer, P. N. (2017, August).
\emph{Stability and variation in student engagement in science classes:
A person-oriented approach}. Paper presented at the Annual Meeting of
the American Psychological Association, Washington, DC.

Beymer, P. N., Rosenberg, J. M., Schmidt, J. A., Naftzger, N.,
Sniegowski, S., Shumow, L. (August, 2017). \emph{Examining relationships
between choice, affect, and engagement in informal STEM programs}. Paper
presented at the Annual Meeting of the American Psychological
Association, Washington, DC.

Greenhalgh, S. P., Rosenberg, J. M., \& Koehler, M. J. (2017, April).
\emph{Combining data sets and methods to explore equity in teacher
professional development. In D. G. Krutka (Chair), Data, big and small}.
Symposium conducted at the meeting of the American Educational Research
Association, San Antonio, TX.

Schmidt, J. A., Rosenberg, J. M., \& Beymer, P. N. (2017, April).
\emph{Momentary engagement profiles: A person-in-context approach to
studying student engagement using experience sampling data}. Paper
presented at the Annual Meeting of the American Educational Research
Association, San Antonio, TX.

Roseth, C. J., Linnenbrink-Garcia, L., Saltarelli, W., Lee, Y-K.,
Rosenberg, J. M. \ldots{} \& Beymer, P. N. (2017, April). \emph{A
design-based intervention on flipped instruction: Longitudinal effects
on undergraduates' engagement and achievement}. Paper presented at the
Annual Meeting of the American Educational Research Association, San
Antonio, TX.

Galey, S., Ferrare, J., \& Rosenberg, J. M. (2017, April). \emph{Idea
brokers, policy convergence, and paradigm shifts: The co-evolution of
school choice and alternative certification issue networks in the
national educational policy discourse, 2000-2015}. Paper presented at
the Annual Meeting of the American Educational Research Association, San
Antonio, TX.

Wright, E., Hao, Q., \& Rosenberg, J. M. (2017, April). \emph{What are
the most important predictors of the earnings of college graduates? Data
from the college scorecard}. Paper presented at the Annual Meeting of
the American Educational Research Association, San Antonio, TX.

Mikeska, J. N., Rosenberg, J. M., Holtzman, S., \& McCaffrey, D. (2017,
April). \emph{Comparing the alignment between two observational measures
of science teachers' instructional practice}. Poster presented at the
National Association for Research in Science Teaching Annual
International Conference, San Antonio, TX.

Greenhalgh, S. P., Rosenberg, J. M., \& Koehler, M. J. (2017, March).
\emph{Avoiding madness in our methods}. Paper presented at the Society
for Information Technology and Teacher Education International
Conference 2017, Austin, TX.

Rosenberg, J. M., Akcaoglu, M., Staudt Willet, K. B., Greenhalgh, S. P.,
\& Koehler, M. J. (2017, March). \emph{A tale of two Twitters:
Synchronous and asynchronous use of the same hashtag}. In P. Resta \& S.
Smith (Eds.), Proceedings of Society for Information Technology \&
Teacher Education International Conference 2017 (pp.~283-286).
Waynesville, NC: Association for the Advancement of Computing in
Education (AACE).

Kessler, A., \& Rosenberg, J. M. (2017, March). \emph{Considering how
teachers' TPACK is leveraged during the mental engineering of
instruction: A theory of action}. Paper presented at the Society for
Information Technology and Teacher Education International Conference
2017, Austin, TX.

Nyland, R., Greenhalgh, S. P., Rosenberg, J. M., Koehler, M. J.,
Veletsianos, G., \& Kimmons, R. (2016, October). \emph{Public data
mining methods, ethics, \& legalities}. Panel presented at the
Association for Educational Communications and Technology International
Convention 2016, Las Vegas, NV.

Rosenberg, J. M., Greenhalgh, S. P., \& Wolf, L. G. (2016, October).
\emph{Participating from near and far: Analyzing online graduate
learning communities with social network analysis}. Paper presented at
the Association for Educational Communications and Technology
International Convention 2016, Las Vegas, NV.

Rosenberg, J. M. (2016, October). \emph{Having agency in conditions that
are not equitable: An examination of Donors Choose data}. Paper
presented at the Association for Educational Communications and
Technology International Convention 2016, Las Vegas, NV.

Phillips, M., Koehler, M. J., \& Rosenberg, J. M. (2016, September).
\emph{Contextualising teachers' TPACK development and enactment}. Paper
presented at the Australian Council for Computers in Education,
Brisbane, Australia.

Rosenberg, J. M. \& Schwarz, C. V. (2016, April). Examining fifth and
sixth grade students' epistemic considerations through an automated
analysis of embedded assessment items. In B. Reiser (Chair),
\emph{Longitudinal studies of elementary and middle school students'
epistemic considerations through participation in scientific practice}.
Related paper set presented at the National Association for Research in
Science Teaching Annual International Conference, Baltimore, MD.
(slides)

Rosenberg, J. M. \& Krist, C. (2016, April). \emph{Characterizing
students' epistemic considerations: An automated computational approach
for embedded assessment responses}. Poster presented at the National
Association for Research in Science Teaching Annual International
Conference, Baltimore, MD. (slides)

Ranellucci, J., Rosenberg, J. M., Klautke, H., Robinson, K. A.,
Saltarelli, W., Linnenbrink-Garcia, L., \& Roseth, C. J. (2016, April).
\emph{Achievement goals, behavioral engagement, and achievement in a
flipped undergraduate anatomy course}. Paper presented at the Annual
Meeting of the American Educational Research Association, Washington,
DC.

Lee, Y.-K., Rosenberg, J. M., Robinson, K. A., Klautke, H., Seals, C.,
Saltarelli, W., Linnenbrink-Garcia, L., \& Roseth, C. J. (2016, April).
\emph{Comparing motivation and achievement in a flipped and traditional
classroom}. Paper presented at the Annual Meeting of the American
Educational Research Association, Washington, DC.

Wormington, S. V., Lee, Y.-K., Seals, C., Rosenberg, J. M., Saltarelli,
W., Roseth, C. J., \& Linnenbrink-Garcia, L. (2016, April).
\emph{Predicting profile permanence: When is motivation stable, why does
it change, and what are the consequences?} Paper presented at the Annual
Meeting of the American Educational Research Association, Washington,
DC.

Ranellucci, J., Robinson, K. A., Rosenberg, J. M., Saltarelli, W.,
Roseth, C. J., \& Linnenbrink-Garcia, L. (2016, April). \emph{Comparing
emotions in-class and during online video lectures in a flipped
classroom}. Paper presented at the Annual Meeting of the American
Educational Research Association, Washington, DC.

Rosenberg, J. M., Ranellucci, J., Lee, Y.-K., Robinson, K., Saltarelli,
W., Linnenbrink-Garcia, L., \& Roseth, C. J. (2016, March).
\emph{Patterns of engagement in a flipped undergraduate anatomy class
and their relations to achievement}. Paper presented at the Society for
Information Technology \& Teacher Education Annual Conference, Savannah,
GA.

Rosenberg, J. M. (2015, November). \emph{Examining what teachers and
researchers discuss at science education conferences through a
computational analysis of Twitter data}. Paper presented at the meeting
of the Association for Educational Communications and Technology,
Indianapolis, IN.

Rosenberg, J. M., Akcaoglu, M., Hamilton, E., Greenhalgh, S. P., \&
Koehler, M. J. (2015, November). \emph{Tweeting U.S.A.: An examination
of State Educational Twitter Hashtags (SETHs)}. Paper presented at the
meeting of the Association for Educational Communications and
Technology, Indianapolis, IN.

Greenhalgh, S. P., Rosenberg, J. M., Keenan, S., \& Koehler, M. J.
(2015, November). \emph{An investigation of the use of digital
portfolios for understanding educators' technology knowledge}. Paper
presented at the meeting of the Association for Educational
Communications and Technology, Indianapolis, IN.

Hamilton, E., Rosenberg, J. M., \& Akcaoglu, M. (2015, November).
\emph{Examining the Substitution Augmentation Modification Redefinition
(SAMR) Model for instructional design and technology integration}. Paper
presented at the meeting of the Association for Educational
Communications and Technology, Indianapolis, IN.

Mehta, R., Rosenberg, J. M., Russo, T., Arnold, B., Marich, H., \& Bell,
J. (2015, November). \emph{A survey of social media use and the effects
of a social media initiative on graduate student engagement}. Paper
presented at the meeting of the Association for Educational
Communications and Technology, Indianapolis, IN.

Rosenberg, J. M., \& Koehler, M. J. (2015, April). Context and
Technological Pedagogical Content Knowledge: A content analysis. In J.
M. Rosenberg \& M. J. Koehler (Chairs), \emph{Addressing the complexity
of teaching with technology: Context and Technological Pedagogical
Content Knowledge}. Symposium conducted at the American Educational
Research Association Annual Meeting, Chicago, IL.

Hamilton, E., Rosenberg, J. M., \& Akcaoglu, M. (2015, April). \emph{The
Substitution Augmentation Modification Redefinition (SAMR) framework for
technology integration: Challenges to its use for guiding K-12 teacher's
pedagogy and practice}. Paper presented at the American Educational
Research Association Annual Meeting, Chicago, IL.

Rosenberg, J. M., Ervin, L., Harris, J., Greenhow, C., Kessler, A., \&
Tai, D. (2015, March). \emph{How should educational technology
researchers consider context? An interactive discussion on context and
teaching and learning with technology}. Panel presented at the meeting
of the Society for Information Technology and Teacher Education
International Conference, Las Vegas, NV.

Akcaoglu, M., \& Rosenberg, J. M. (2015, March). \emph{Best practices
for designing synchronous and asynchronous online teaching for adult
learners}. Poster presented at the meeting of the Society for
Information Technology and Teacher Education, Las Vegas, NV.

Rosenberg, J. M., Schwarz, C. V., \& Lee, S. W.-Y., \& Akcaoglu, M.
(2015, April). A comparative longitudinal case study of the use of
scientific modeling in the pedagogical practice of two fifth-grade
science teachers. In A. Lo (Chair), \emph{Leveraging the epistemic
dimensions of scientific practice to support students' meaningful
engagement in modeling}. Related paper set presented at the National
Association for Research in Science Teaching Annual International
Conference, Chicago, IL.

Rosenberg, J. M., Schwarz, C.V., Akcaoglu, M., \& Lee, S.W-Y. (2014,
October). \emph{Comparative longitudinal case studies of two middle
school teachers' use of scientific modeling}. Poster presented at the
Advances in Educational Psychology Conference. Fairfax, VA.

Lee, M., Schwarz, C. V., Ke, L., \& Rosenberg, J. M. (2014, April).
\emph{Analyzing fifth-grade students' engagement in scientific modeling:
Changes in students' epistemologies-in-practice over time}. Paper
presented at the meeting of the National Association for Research in
Science Teaching, Philadelphia, PA.

Ke, L., Schwarz, C. V., Lee, M. \& Rosenberg, J. M. (2014, April).
\emph{Examining elementary students' attention to mechanism as they
engage in scientific modeling across content areas}. Paper presented at
the meeting of the National Association for Research in Science
Teaching, Philadelphia, PA.

Koehler, M. J., Rosenberg, J. M., Greenhalgh, S., Zellner, A. L., \&
Mishra, P. (2014, March). Analyzing students' portfolios for the
development of TPACK. In J. Voogt (Chair), \emph{Artifacts demonstrating
teachers' technology integration competencies}. Symposium presented at
the meeting of the Society for Information Technology and Teacher
Education, Jacksonville, FL.

\subsubsection{Invited Talks}\label{invited-talks}

Rosenberg, J.M. (September, 2019). \emph{Data science and STEM
education}. Presentation at the Middle Tennessee State University
Mathematics and Science Education Doctoral Seminar series. Middle
Tennessee State University, Murfreesboro, TN.

Rosenberg, J. M. (February, 2019). \emph{Making sense of recent advances
in the Technological Pedagogical Content Knowledge framework}. English
International Congress at the Universidad Tecnical de Norte, Ibarra,
Ecuador.

Rosenberg, J. M. (March, 2016). \emph{An introduction to R for
programming and statistical analysis in education}. Georgia Southern
University College of Education, Statesboro, GA.

\subsubsection{Other Presentations}\label{other-presentations}

Camponovo, M., Lawson, M. A., \& Rosenberg, M. J. (July, 2019).
Integrating geospatial tech with math and science pre-service teachers.
2019 Education Summit @ ESRI UC. San Diego, CA.

Jones, R. S., \& Rosenberg, J. M. (February, 2019). Latent class
modeling of whole class discussions about data, statistics, and
probability. Presentation at the 13th Annual Tennessee STEM Education
Research Conference. Murfreesboro, TN.

Lawson, M., Rosenberg, J. M., \& Camponovo, M. (February, 2019). Better
together? Findings from a combined, integrated STEM unit with
pre-service mathematics and science teachers. Presentation at the 13th
Annual Tennessee STEM Education Research Conference. Murfreesboro, TN.

\subsection{Competitive Research
Training}\label{competitive-research-training}

New Faculty Mentoring Program, AERA Division C, 2019

Graduate Student Seminar, AERA Division C, 2016

Early Career Seminar, National Science Foundation and AECT, 2015

Advanced Training Institute on Research Methods with Diverse Groups,
American Psychological Association, 2014

\subsection{Research Experience}\label{research-experience}

Research Assistant, Profiles of Science Engagement, Jennifer Schmidt
(PI), 2017-2018

Research Assistant, Self-Generated Research Experiences to Support
Biomedical/Behavioral Research Careers, Lisa Linnenbrink-Garcia (PI),
2017

Research Assistant, STEM Interest and Engagement, Jennifer Schmidt (PI),
2016-2018

Research Assistant, Supporting scientific practices in elementary and
middle school classrooms, Christina Schwarz (MSU PI), 2013-2017

\subsection{Software Developed}\label{software-developed}

\subsubsection{R packages on Comprehensive R Archive Network
(CRAN)}\label{r-packages-on-comprehensive-r-archive-network-cran}

Rosenberg, J. M., van Lissa, C. J., Beymer, P. N., Anderson, D. J.,
Schell, M. J. \& Schmidt, J. A. (2019). tidyLPA: Easily carry out Latent
Profile Analysis (LPA) using open-source or commercial software {[}R
package{]}. \url{https://data-edu.github.io/tidyLPA/}

KonFound-It! R package: Rosenberg, J. M., Xu, R., \& Frank, K. A.
(2019). konfound: Quantify the robustness of causal inferences {[}R
package{]}. \url{https://jrosen48.github.io/konfound/}

Rosenberg, J. M., Schmidt, J. A., Beymer, P. N., \& Steingut, R. (2018).
prcr: Person-Centered Analysis {[}R package{]}.
\url{https://CRAN.R-project.org/package=prcr}

Rosenberg, J. M., \& Lishinski, A. (2018). clustRcompaR: Easy interface
for clustering a set of documents and exploring group-based patterns
{[}R package{]}. \url{https://github.com/alishinski/clustRcompaR}

\subsubsection{Interactive Web
Applications}\label{interactive-web-applications}

Rosenberg, J. M., Xu, R., \& Frank, K. A. (2018). Konfound-It!: Quantify
the Robustness of Causal Inferences. \url{http://konfound-it.com}.

\subsection{Teaching}\label{teaching}

\subsubsection{Teaching Awards}\label{teaching-awards}

\textbf{Michigan State University}

MSU-AT\&T Instructional Technology Award: Best Online Course, 2014

MSU-AT\&T Instructional Technology Award (Honorable Mention): Best
Online Course, 2013

\subsubsection{Courses Taught}\label{courses-taught}

Instructor at the University of Tennessee, Knoxville

\emph{Nature of Mathematics and Science Education} (SCED 572, M.A.~and
Ph.D.~class)\\
\emph{Teaching Science in Grades 7-12} (TPTE 495, SCED 496, \& SCED 543,
B.S. \& M.A.~class)

Instructor at Michigan State University:

\emph{Psychology of Learning in School and Other Settings} (CEP 800,
M.A.~class)\\
\emph{Approaches to Educational Research} (CEP 822, M.A.~class)\\
\emph{Technology and Leadership} (CEP 815, M.A.~class)

Teaching Assistant at Michigan State University:

\emph{Proseminar in Educational Psychology and Educational Technology}
(CEP 900, Ph.D.~class)\\
\emph{Proseminar in Educational Technology} (CEP 807 / ED 870,
M.A.~class)\\
\emph{Educational Inquiry} (CEP 900, Ph.D.~class)\\
\emph{Social-Emotional Development Across the Lifespan} (CEP 904,
Ph.D.~class)

\subsection{Service}\label{service}

\subsubsection{Editorial Service}\label{editorial-service}

Editorial Review Board Member, \emph{Journal of Research in Science
Teaching}, 2019-2022

Editorial Review Board Member, \emph{Contemporary Issues in Technology
and Teacher Education (Science Education Section)}, 2019 - Present

Editorial Review Board Member, \emph{Journal of Research on Technology
in Education}, 2016 - Present

Special Issue Editor, \emph{Australasian Journal of Educational
Technology}, 2017

\subsubsection{Service to the
Profession}\label{service-to-the-profession}

American Educational Research Association, Division C, Section 1D:
Science Program Co-Chair, 2019-2020

National Science Foundation, STEM+C review planel, 2019

Member, Technological Pedagogical Content Knowledge (TPACK) Special
Interest Group (SIG) Award Committee, 2019

Co-chair, TPACK SIG, Society for Information Technology and Teacher
Education , 2015-2017

Membership Committee, Division 15 (Educational Psychology), American
Psychological Association (APA), 2014-2017

Communications Deputy, Division C, American Educational Research
Association, 2015-2016

Associate Chair, TPACK SIG, Society for Information Technology and
Teacher Education, 2014-2015

\subsubsection{Conference Review
Activity}\label{conference-review-activity}

Reviewer, National Association for Research in Science Teaching Annual
Conference, 2019

Review Panel Member, American Educational Research Association (AERA)
Annual Meeting, 2015-2019

Reviewer, Association for Science Teacher Education Annual Conference,
2019

Program Committee Member, International Conference on Computer-Supported
Collaborative Learning, 2017

Graduate Student Reviewer, American Educational Research Association
(AERA) Annual Meeting, 2014

Reviewer, Association for Educational Communications and Technology
(AECT) International Convention, 2016

Reviewer, American Psychological Association (APA) Convention, 2015

\subsubsection{Service to the Community}\label{service-to-the-community}

Reviewer, Proposals from Knox County Schools students for theNASA
Student Spaceflight Experiment Program

\subsubsection{Ad-hoc Journal Article
Reviews}\label{ad-hoc-journal-article-reviews}

AERA Open, Education Sciences (2), Journal of Open Source Education,
Journal of Research in Science Teaching, TechTrends, 2019

Contemporary Educational Psychology, Computers \& Education,
Australasian Journal of Educational Technology (2), Journal of Open
Source Software, Asia-Pacific Education Researcher, 2018

Computers \& Education, Journal of Educational Technology \& Society,
2017

Computers \& Education, British Journal of Educational Technology,
E-Learning and Digital Media (2), 2016

Contemporary Issues in Technology and Teacher Education, 2015

\subsubsection{Departmental Service}\label{departmental-service}

\emph{University of Tennessee, Knoxville}

Committee member for one Ph.D student (2019)

Committee member for five Master's degree students (2019)

\emph{Michigan State University}

Member of two practicum committees for Educational Psychology and
Educational Technology program Ph.D.~students, Michigan State
University, 2014-2018

Search Committee Member, Program Specialist, Master of Arts in
Educational Technology Program, Michigan State University, 2015

\subsubsection{Service to the
Community}\label{service-to-the-community-1}

Reviewer, Proposals from Knox County Schools students to the NASA
Student Spaceflight Experiment Program, 2018

Provided enrichment activities related to data science to two classes,
Knox County Schools, Private School in Knox County, 2018

\subsubsection{Workshops and Outreach}\label{workshops-and-outreach}

Anderson, D. J., and Rosenberg, J. M. (2019, April). Transparent and
reproducible research with R. Workshop carried out at the Annual Meeting
of the American Educational Research Association, Toronto, Canada.

Rosenberg, J. M. (2017, April). Introduction to R for Data Analysis.
Presentation at the School of Criminal Justice, Michigan State
University.

Ranellucci, J., \& Rosenberg, J. M. (2016, February). Motivating our
students: A partnership between Michigan Virtual Schools and Michigan
State University. Workshop at Michigan Virtual University, East Lansing,
MI.

Rosenberg, J. M. (2014, April). Action research with mobile devices and
other ``disruptive'' technologies. Presentation at the Best of the
Michigan Association for Computer Users in Learning Conference,
Waterford, MI.

Rosenberg, J. M. (2014, February). Action research with mobile devices.
Presentation at the Michigan Association for Computer Users in Learning
Mobile Learning Conference, Kalamazoo, MI.

Sawaya, S., \& Rosenberg, J. M (2014, February). Master of Arts in
Educational Technology Mobile Learning Workshop. Workshop at Michigan
State University, East Lansing, MI.

\subsubsection{Campus and Departmental
Presentations}\label{campus-and-departmental-presentations}

Rutherford, T., \& Rosenberg, J. M. (2019, February). \emph{Motivational
correlates of choice after failure within an elementary mathematics
software}. Presentation at the NC State College of Education Celebration
of Research.

Rosenberg, J. M. (2019, January). *Engaging students in science:
Findings from an experience sampling method approach. Presentation at
the East Tennessee STEM Hub Crossing Boundaries for STEM Teaching
regional meeting and mini-conference. Knoxville, TN.

Rosenberg, J. M., Beymer, P. N., \& Schmidt, J. A. (2017, February).
Does choosing the problem or topic matter? Using a person-in-context
approach to understand student engagement in science. Poster presented
at the Create4Stem MiniConference 2017, East Lansing, MI.

Rosenberg, J. M. (2016, April). Momentary engagement profiles: An
examination of student engagement in science settings using experience
sampling methodology. Presentation at the Michigan State University
Educational Psychology and Educational Technology Program Informal
Colloquium, East Lansing, MI.

Rosenberg, J. M., \& Schwarz, C. V. (2016, February). Examining the
development of fifth and sixth grade students' epistemic considerations
over time through an automated analysis of embedded assessment items.
Poster presented at the Create4Stem MiniConference 2016, East Lansing,
MI.

Rosenberg, J. M. (2015, September). Achievement goals, in- and
out-of-class engagement, and students' achievement in a flipped
undergraduate anatomy class. Presentation at the Michigan State
University Educational Psychology and Educational Technology Program
Informal Colloquium, East Lansing, MI.

Rosenberg, J. M., Akcaoglu, M., Schwarz, C.V., \& Lee, S.W-Y. (2015,
February). Comparative longitudinal case studies of two middle school
teachers' use of scientific modeling. Poster presented at the
Create4Stem MiniConference 2015, East Lansing, MI.

Lee, M., Schwarz, C.V., Ke, L., Rosenberg, J. M., Reiser, B., Berland,
L., Kenyon, L., Wilson, M., Draney, K. (2015, February). Epistemic
considerations in scientific practices for elementary \& middle schools.
Poster presented at the Create4Stem MiniConference 2015, East Lansing,
MI.

Wolf, L. G., Henriksen, D., Sawaya, S., \& Rosenberg, J. M. (2014,
December). EdCamp with Team MAET. Presentation at the Michigan State
University Master of Arts in Educational Technology Bridge Webinar
Series, East Lansing, MI.

Rosenberg, J. M. (2014, November). Integrating ``disruptive''
technologies into teaching with action research and Technological
Pedagogical Content Knowledge (TPACK). Presentation at the Michigan
State University Educational Technology Conference, East Lansing, MI.

Wolf, L. G., Henriksen, D., Sawaya, S., \& Rosenberg, J. M. (2014,
March). Mobile learning for educators. Presentation at the Michigan
State University Master of Arts in Educational Technology Bridge Webinar
Series, East Lansing, MI.

Rosenberg, J. M. (2014, February). Context and Technological Pedagogical
Content Knowledge: Preliminary results of a content analysis.
Presentation at the Michigan State University Educational Psychology and
Educational Technology Program Informal Colloquium, East Lansing, MI.

Ke, L., Lee, M., Rosenberg, J. M., \& Schwarz, C.V. (2014, February).
Modeling across content areas: Examining elementary students' attention
to mechanism. Poster presented at the Create4Stem MiniConference 2014,
East Lansing, MI.

Rosenberg, J. M., Rapa, L., \& Wolf, L. G. (2013, February). CEP 815 and
the transition from ANGEL to Desire2Learn. Poster presented at the 6th
Annual Faculty Technology Showcase.

Rosenberg, J. M. (2012, November). Mobile learning for teachers.
Presentation at the Michigan State University Educational Technology
Conference, East Lansing, MI.

\subsection{Consulting}\label{consulting}

2017-2019, Statistical software development Kenneth Frank, Michigan
State University

2017, Statistical analysis\\
Yael Shwartz, Weizmann Institute

2016, Statistical analysis\\
Lara Kassab, San Jose State University

\subsection{Professional Affiliations}\label{professional-affiliations}

American Educational Research Association, 2012 - Present\\
Association for the Advancement of Computing in Education, 2012 -
Present\\
Association for Science Teacher Education, 2018 - Present\\
National Association for Research in Science Teaching, 2015 - Present


\end{document}
